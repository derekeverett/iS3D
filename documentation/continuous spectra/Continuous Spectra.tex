%\documentclass[showpacs,aps,prd,nofootinbib,showkeys,superscriptaddress]{revtex4-1}
\documentclass[10.5pt,aps,prd,superscriptaddress]{revtex4}
%%%%%%%%%%%%%%%%%%%%%%%%%%%%%%%%%%%%%%%%%%%%%%%%%%%%%%%%%%%%%%%%%%%%%%%%%
%packages
\usepackage{graphicx}
\usepackage{bm}
\usepackage{amssymb}
\usepackage{amsmath,latexsym}
\usepackage[usenames]{color}
\usepackage{subfigure}
\usepackage{subfigure}
\usepackage{physymb}
\usepackage{slashed}
\usepackage{multirow,array}
\usepackage{mathtools}
\usepackage{mathrsfs}
\usepackage[colorlinks=false,linktocpage=true]{hyperref}
\usepackage{hyperref}
\usepackage[utf8]{inputenc}
\usepackage{lipsum}
\usepackage{dsfont}
%
%\renewcommand{\baselinestretch}{1.5}
%
%
\usepackage{soul}
\usepackage{color}
\usepackage[colorlinks=false,linktocpage=true]{hyperref}
\usepackage{hyperref}
%\usepackage[retainorgcmds]{IEEEtrantools}
%%%%%%%%%%%%%%%%%%%%%%%%%%%%%%%%%%%%%%%%%%%%%%%%%%%%%%%%%%%%%%%%%%%%%%%%
\newcommand{\be}{\begin{equation}}
\newcommand{\ee}{\end{equation}}
\newcommand{\bea}{\begin{eqnarray}}
\newcommand{\eea}{\end{eqnarray}}
\newcommand{\idd}{\indent \indent}
\newcommand{\blue}{\textcolor{blue}}
\newcommand{\red}{\textcolor{red}}
%%%%%%
\newcommand{\eq}{\text{eq}}
\newcommand{\api}{\big(b_n \, \rho^{(n)}_B\big)}
\newcommand{\bpi}{\big(b_n \, \epsilon^{(n)}_{eq}\big)}
\newcommand{\cpi}{(\rho_B T)}
\newcommand{\dpi}{\big(b_n \, \epsilon^{(n)}_{eq}\big)}
\newcommand{\epi}{\mathcal{I}_{30}}
\newcommand{\fpi}{\mathcal{I}_{31}}
\newcommand{\gpi}{(\rho_B T)}
\newcommand{\hpi}{\mathcal{I}_{31}}
\newcommand{\jpi}{\mathcal{I}_{32}}
\newcommand{\bmu}{\langle \mu \rangle}
\newcommand{\bnu}{\langle \nu \rangle}
\newcommand{\munu}{{\mu\nu}}
%%%%%%%
\newcommand{\ds}{\delta s}
\newcommand{\Kn}{\text{Kn}}
\newcommand{\del}{\partial}
\newcommand{\tr}{\tau_{r}}
\newcommand{\feq}{f_{\text{eq}}}
\newcommand{\first}{1^{\text{st}}}
\newcommand{\pxp}{(- p \cdot \Xi \cdot p)}
\newcommand{\px}{p^{\langle\mu\rangle}}
\newcommand{\dft}{\delta\tilde{f}}
\newcommand{\ddft}{\delta \dot{\tilde{f}}}
\newcommand{\n}{\newline}
\newcommand{\up}{u \cdot p}
\newcommand{\aP}{\alpha_\perp}
\newcommand{\aL}{\alpha_L}
\newcommand{\aPsq}{\alpha^2_\perp}
\newcommand{\aLsq}{\alpha^2_L}
\newcommand{\ppmunu}{p^{\{\mu} p^{\nu\}}}
\newcommand{\zp}{z \cdot p}
\newcommand{\mzp}{(- z \cdot p)}
\newcommand{\pOp}{(p \cdot \Omega \cdot p)}
\newcommand{\Pm}{\mathcal{P}}
\newcommand{\ene}{\mathcal{E}}
\newcommand{\alphavec}{\mathbf{\alpha}}
\newcommand{\alphaT}{\alpha_\perp}
\newcommand{\alphaL}{\alpha_L}
\newcommand{\pbar}{\bar{p}}
\newcommand{\mbar}{\bar{m}}
\newcommand{\order}{\mathcal{O}}
\newcommand{\BL}{\beta_\Lambda}
\newcommand{\PL}{\mathcal{P}_L}
\newcommand{\Pperp}{\mathcal{P}_\perp}
\newcommand{\Peq}{\mathcal{P}_\text{eq}}
\newcommand{\bs}{\begin{subequations}}
\newcommand{\es}{\end{subequations}}
\newcommand{\beal}{\begin{align}}
\newcommand{\enal}{\end{align}}
\newcommand{\pperp}{p^{\{\mu\}}}
\newcommand{\pperpnu}{p^{\{\nu\}}}
\newcommand{\nabperp}{\tilde{\nabla}}
\newcommand{\D}{\mathcal{D}}
\newcommand{\M}{\mathcal{M}}
\newcommand{\R}{\mathcal{R}}
\newcommand{\Hm}{\mathcal{H}}
\newcommand{\J}{\mathcal{J}}
\newcommand{\N}{\mathcal{N}}
\newcommand{\A}{\mathcal{A}}
\newcommand{\B}{\mathcal{B}}
\newcommand{\Z}{\mathcal{Z}}
\newcommand{\T}{\mathcal{T}}
\newcommand{\LRF}{\text{LRF}}
\newcommand{\piperp}{\pi_\perp}
\newcommand{\Wperp}{W_{\perp z}}
\newcommand{\nn}{\newline\newline}
%\newcommand{\order}{\mathcal{O}}
%%%%%%%%%%%%%%%%%%%%%%%%%%%%%%%%%%%%%%%%%%%%%%%%%%%%%%%%%%%%%%%%%%%%%%%%
\begin{document}
\title{Particle Sampler in iS3D}
\maketitle
Right now, what we have implemented in the particle sampler code so far is the sampling of an ideal distribution for each particle species:
\be
 f_{i,\mathrm{eq}}(x,p) = g_i \left[\exp\left(\frac{p \cdot u(x) - b_i \,\mu_B(x)}{T(x)}\right) + \Theta_i\right]^{-1}
\ee
%
where $g_i$ is the spin degeneracy and $\Theta_i = (1,-1,0)$ for fermions, bosons, and Boltzmann particles. This simplification will be relaxed later when we include viscous corrections from the linearized $\delta f$ or modified equilibrium distribution $f_{\eq}^{\text{(mod)}}$. 
\nn
Here we outline the steps for sampling particles from the Cooper Frye formula:
\be
\label{eq1}
  dN_i(x,p) = \frac{p \cdot d^3\sigma(x)}{(2\pi\hbar)^3} \, \frac{d^3p}{E} \, f_{i,\eq}(x,p) 
\ee
It mostly follows the sampling procedure from Long-Gang's CLVisc paper (Phys. Rev. C 97, 064918)
\nn
\nn
1) For each freezeout cell, compute the mean number of hadrons 
\newline
\be 
\Delta N = \sum_i \frac{g_i}{(2\pi\hbar)^3}\int \frac{d^3p}{E} \, p \cdot d^3\sigma \, f_{i,\eq}(u\cdot p)  = u \cdot d^3\sigma \sum_i n_{i,\eq}
\ee
where 
\be
n_{i,\eq} = \frac{g_i}{(2\pi\hbar)^3} \int d^3p_{\text{LRF}} f_{i,\eq}(u\cdot p) = \frac{g_i}{2\pi^2\hbar^3} \int_0^\infty  \frac{p^2_{\text{LRF}} \,  dp_{\text{LRF}} }{\exp\left[\dfrac{\sqrt{p^2_{\text{LRF}} + m_i^2}- b_i \,\mu_B(x)}{T(x)}\right] + \Theta_i}
\ee
is the LRF equilibrium number density of each species and $p_{\text{LRF}}$ is the LRF radial momentum. To compute radial momentum integral, we expand the denominator as a geometric series, keeping only the first term for heavy hadrons and first 10 terms for pions (alternatively, you could do Gauss-quadrature as a future option).
\nn
We skip the rest of the sampling procedure for any cells with $u \cdot d^3\sigma < 0$ ($\Delta N < 0$), which rarely occurs. 
\nn
\nn
2) Then sample the number of hadrons $N$ in that cell with a Poisson distribution
\newline
\be
P(N) =  \frac{(\Delta N)^{N}\exp(-\Delta N)}{N!}
\ee
This assumes the number of each species can be individually sampled with a Poisson distribution.
\nn
\nn
3) Out of $N$ hadrons, sample the number of each species $N_i$ with a discrete probability distribution
\newline
\be
P(N_i) =\frac{\Delta N_i}{\Delta N}
\ee 
\nn
\nn
4) For each sampled particle, sample its LRF momentum $p^\mu_{\text{LRF}} = (u\cdot p, -X \cdot p, - Y \cdot p, - Z \cdot p)$ from the Cooper Frye formula (2). We further assume the equilibrium distribution (1) is Boltzmann-like or $\Theta_i = 0$ (will relax for pions in the immediate future). We outline the momentum sampling procedure for light and heavy particles separately.
\nn
{\bf{Light Boltzmann particles $\bf{(T / m > 0.6)}$:}}
\nn
The Cooper Frye formula (2) in spherical coordinates takes the form (drop constant factors) 
\be
 dN_i \sim \frac{p \cdot d^3\sigma}{u\cdot p} \exp\left(\frac{p_{\text{LRF}} - u \cdot p }{T}\right) \, \left[p_{\text{LRF}}^2 \, \exp\left(-\frac{p_{\text{LRF}}}{T}\right) \, dp _{\text{LRF}}\, d\phi_{\text{LRF}} \, d\cos\theta_{\text{LRF}}\right]
\ee
First we draw $p^\mu_\LRF$ from the massless Boltzmann distribution in the brackets [...] in Eq.\,(7). This is done with Scott Pratt's trick: one introduces the coordinates $(r_1, r_2, r_3) \in (0,1]$ 
\bs
\beal
p_\LRF &= - T (\log r_1 + \log r_2 + \log r_3) \\
\phi_\LRF & = 2\pi \left(\frac{\log r_1 + \log r_2}{\log r_1 + \log r_2 + \log r_3}\right)^2 \\
\cos\theta_\LRF & = \frac{\log r_1 - \log r_2}{\log r_1 + \log r_2}
\end{align}
\es
The determinant of the Jacobian is $\abs{J} = \big|\dfrac{\partial p}{\partial r}\big| = \dfrac{8 \pi T^3 \exp(p_\LRF / T)}{p_\LRF^2}$. The massless distribution reduces to
\be
p_{\text{LRF}}^2 \, \exp\left(-\frac{p_{\text{LRF}}}{T}\right) \, dp _{\text{LRF}}\, d\phi_{\text{LRF}} \, d\cos\theta_{\text{LRF}} = 8\pi T^3 dr_1 dr_2 dr_3
\ee
This means $(r_1,r_2,r_3)$ can be sampled uniformly. After we have sampled the $r$-coordinates, we get the spherical coordinates using Eq.\,(8). Then, we can compute the components of the proposed $p^\mu_\LRF$
\bs
\beal
E_\LRF &= u \cdot p =  \sqrt{p_\LRF^2 + m_i^2} \\
p^x_\LRF &= p_\LRF \sin\theta_\LRF \cos\phi_\LRF \\ 
p^y_\LRF &= p_\LRF \sin\theta_\LRF \sin\phi_\LRF \\ 
p^z_\LRF &= p_\LRF \cos\theta_\LRF  
\end{align}
\es
The Cooper Frye formula (7) still has an additional weight factor $w_{\text{light}}$ outside the brackets [...], which is now constant. To make $0 \leq w_{\text{light}} \leq 1$, we rescale it by a constant factor. In this case, we use the Minkowski inequality:
\be
\abs{p \cdot d^3\sigma} \leq (u\cdot p)\left(\abs{u \cdot d^3\sigma} + \sqrt{(u \cdot d^3\sigma)^2 - d^3\sigma \cdot d^3\sigma}\right)
\ee
The weight now becomes
\be
w_{\text{light}} = \frac{\abs{p \cdot d^3\sigma} \,\exp\left[\dfrac{p_{\text{LRF}} - u \cdot p }{T}\right]}{(u\cdot p) \left(\abs{u \cdot d^3\sigma} + \sqrt{(u \cdot d^3\sigma)^2 - d^3\sigma \cdot d^3\sigma}\right)
}
\ee
which lies between 0 and 1. The dot product $\abs{p \cdot d^3\sigma}$ is computed in the LRF. 
\nn
Finally, we draw a random number $m \in [0,1]$ from a uniform distribution. If $m < w_{\text{light}}$, then $p^\mu_\LRF$ is accepted. If rejected, we repeat the above steps (7) - (12) until it gets accepted. \blue{(I feel like there's a difference from my version and Long-Gang's version in the weight step)}
\nn
\nn
\nn
{\bf{Heavy Boltzmann particles $\bf{(T / m < 0.6)}$:}}
\nn
For heavy Boltzmann particles, we replace the radial momentum $p_\LRF$ with the variable $k = E_\LRF - m$, to rewrite the Cooper Frye formula (2) as (drop out constant factors) 
\be
 dN_i \sim \frac{p \cdot d^3\sigma}{u\cdot p} \, \frac{p_\LRF}{E_\LRF} \, \left[(m^2 + 2mk + k^2) \, \exp\left(-k/T\right) \, dk\, d\phi_{\text{LRF}} \, d\cos\theta_{\text{LRF}}\right]
\ee
Then we sample from either one of the three distributions in the parentheses (...) based on their $k$-integrated weights. 
\bs
\beal
I_1 &= \int_0^\infty dk \, m^2 \, \exp(-k/T) = m^2 T \\
I_2 &= \int_0^\infty dk \, 2mk \, \exp(-k/T) = 2 m T^2 \\
I_3 &= \int_0^\infty dk \, k^2 \, \exp(-k/T) = 2 T^3 \\ 
I_{\text{tot}} &= I_1 + I_2 + I_3
\end{align}
\es
The probability of sampling from the first distribution is approximately $I_1 / I_{\text{tot}}$, etc. We draw a random number $q \in [0,1]$ from a uniform distribution and look at each of the possible outcomes: 
\nn
{\bf{Case 1: $\bf{e^{-k/T}}$}}
\nn
If $0 \leq q \leq I_1 / I_{\text{tot}}$ (most likely scenario for heavy hadrons), we draw $p^\mu_\LRF$ from the distribution
\be
\exp\left(-k/T\right) \, dk\, d\phi_{\text{LRF}} \, d\cos\theta_{\text{LRF}}
\ee
To sample $k$, we only need to sample one $r$-coordinate $r_1 \in (0,1]$, where
\be
k = - T \log r_1
\ee
One can easily check that $\exp(-k/T) \,dk = T dr_1$, so we can sample $r_1$ uniformly. Once we sample $k$, $E_\LRF$ and $p_\LRF$ are determined. Next, the angles $(\phi_\LRF, \cos\theta_\LRF)$ are sampled uniformly. The $p^\mu_\LRF$ components can then be computed like before, and we check its acceptance against the weight factor
\be
w_{\text{heavy}}= \frac{\abs{p \cdot d^3\sigma} \, p_\LRF / E_\LRF}{(u\cdot p) \left(\abs{u \cdot d^3\sigma} + \sqrt{(u \cdot d^3\sigma)^2 - d^3\sigma \cdot d^3\sigma}\right)
}
\ee 
If rejected, repeat the steps for {\bf{Case 1}} until accepted. 
\nn\nn
{\bf{Case 2: $\bf{k \, e^{-k/T}}$}}
\nn
If $I_1 / I_{\text{tot}} < q \leq (I_1+I_2) / I_{\text{tot}}$, then we draw $p^\mu_\LRF$ from the distribution 
\be
k \exp\left(-k/T\right) \, dk\, d\phi_{\text{LRF}} \, d\cos\theta_{\text{LRF}}
\ee
For this case, we need to draw $(k,\phi_\LRF)$ by sampling two variables $(r_1,r_2) \in (0,1]$, where
\bs
\beal
k &= - T (\log r_1 + \log r_2)  \\
\phi_\LRF &= 2 \pi \frac{\log r_2}{\log r_1 + \log r_2}
\end{align}
\es
One can check that $k \exp(-k/T) \, dk \, d \phi_\LRF = 4\pi T^2 dr_1 dr_2$, so we can sample $(r_1, r_2)$ uniformly. The angle $\cos\theta_\LRF$ is sampled separately. Then compute $p^\mu_\LRF$ and check its acceptance against the weight $w_{\text{heavy}}$ in (17). If rejected, repeat the steps for {\bf{Case 2}} until accepted. 
\nn
{\bf{Case 3: $\bf{k^2 \, e^{-k/T}}$}}
\nn
If $(I_1 + I_2) / I_{\text{tot}} < q \leq 1$, then we draw $p^\mu_\LRF$ from the distribution 
\be
k^2 \exp\left(-k/T\right) \, dk\, d\phi_{\text{LRF}} \, d\cos\theta_{\text{LRF}}
\ee
The sampling procedure is the same as the massless distribution [...] in (7) after one swaps out $p_\LRF$ for $k$. After computing $p^\mu_\LRF$, check its acceptance against $w_{\text{heavy}}$ in (17). If rejected, repeat the steps for {\bf{Case 3}} until accepted. 
\nn
5) After sampling the particle's LRF momentum, compute the lab frame momentum in Milne coordinates by using the decomposition:
\newline
\be
p^\mu = (u \cdot p) u^\mu + (-X \cdot p)X^\mu + (-Y \cdot p)Y^\mu + (-Z \cdot p)Z^\mu 
\ee
The components of the spatial basis vectors are
\bs
\beal
X^\mu &= \left(u_\perp \cosh\kappa_L, \frac{u^x \, u^\tau_\perp}{u_\perp}, \frac{u^y \, u^\tau_\perp}{u_\perp}, \frac{u_\perp \sinh\kappa_L}{\tau} \right) \\
Y^\mu &=\left(0, - \frac{u^y}{u_\perp}, \frac{u^x}{u_\perp},0\right) \\
Z^\mu &= \left(\sinh\kappa_L, 0, 0, \frac{\cosh\kappa_L}{\tau}\right)
\end{align}
\es
where $\kappa_L = \tanh^{-1}(\tau u^\eta / u^\tau)$, $u_\perp = \sqrt{(u^x)^2 + (u^y)^2}$, and  $u^\tau_\perp = \sqrt{1+u_\perp^2}$. We also compute the Cartesian lab frame energy $E$ and longitudinal momentum $p^z$.
\bs
\beal
E &= p^\tau \cosh\eta + \tau p^\eta \sinh\eta \\
p^z &= \tau p^\eta \cosh\eta + p^\tau \sinh\eta
\end{align}
\es
\nn
6) If $p \cdot d^3 \sigma \geq 0$, amend the sample particle's MC ID number, position and momentum to the output particle list file (current file format):
\newline
\be \nonumber
(\text{MC}, \indent \tau, \indent x, \indent y, \indent \eta, \indent E, \indent p^x, \indent p^y, \indent p^z)
\ee
Otherwise, we discard it. 
\end{document}







